%%%%%%%%%%%%%%%%%%%%%%%%%%%%%%%%%%%%%%%%%%%%%%%%%%%%%%%%%%%%%%%%%%
%%%%%%%% ICML 2015 EXAMPLE LATEX SUBMISSION FILE %%%%%%%%%%%%%%%%%
%%%%%%%%%%%%%%%%%%%%%%%%%%%%%%%%%%%%%%%%%%%%%%%%%%%%%%%%%%%%%%%%%%

% Use the following line _only_ if you're still using LaTeX 2.09.
%\documentstyle[icml2015,epsf,natbib]{article}
% If you rely on Latex2e packages, like most moden people use this:
\documentclass{article}

% use Times
\usepackage{times}
% For figures
\usepackage{graphicx} % more modern
%\usepackage{epsfig} % less modern
\usepackage{subfigure} 

% For citations
\usepackage{natbib}

% For algorithms
\usepackage{algorithm}
\usepackage{algorithmic}

% As of 2011, we use the hyperref package to produce hyperlinks in the
% resulting PDF.  If this breaks your system, please commend out the
% following usepackage line and replace \usepackage{icml2015} with
% \usepackage[nohyperref]{icml2015} above.
\usepackage{hyperref}

% Packages hyperref and algorithmic misbehave sometimes.  We can fix
% this with the following command.
\newcommand{\theHalgorithm}{\arabic{algorithm}}

% Employ the following version of the ``usepackage'' statement for
% submitting the draft version of the paper for review.  This will set
% the note in the first column to ``Under review.  Do not distribute.''
%\usepackage{icml2015} 

% Employ this version of the ``usepackage'' statement after the paper has
% been accepted, when creating the final version.  This will set the
% note in the first column to ``Proceedings of the...''
\usepackage[accepted]{icml2015}


% The \icmltitle you define below is probably too long as a header.
% Therefore, a short form for the running title is supplied here:
\icmltitlerunning{Submission and Formatting Instructions for ICML 2015}

\begin{document} 

\twocolumn[
\icmltitle{Exploring the Naturalness of Buggy Code with Recurrent Neural Networks)}

% It is OKAY to include author information, even for blind
% submissions: the style file will automatically remove it for you
% unless you've provided the [accepted] option to the icml2015
% package.
\icmlauthor{Jack Lanchantin}{jjl5sw@virginia.edu}
\icmladdress{University of Virginia, Department of Computer Science}
\icmlauthor{Ji Gao}{email@coauthordomain.edu}
\icmladdress{University of Virginia, Department of Computer Science}

% You may provide any keywords that you 
% find helpful for describing your paper; these are used to populate 
% the "keywords" metadata in the PDF but will not be shown in the document
\icmlkeywords{boring formatting information, machine learning, ICML}

\vskip 0.3in
]

\begin{abstract} 
Statistical language models are powerful tools which have been used for many tasks within natural language processing.
Recently, they have been used for other sequential data such as source code. \cite{ray2015naturalness} 
\end{abstract} 

\section{Introduction}
\label{introduction}
Natural language is inherently very well understood by humans. There are certain linguistics and structures associated with natural language which make it fluid and efficient. These repetitive and predictive properties of natural language make it easy to exploit via statistical language models. Although the actual semantics are very much different, source code is also repetitive and predictive. Some of this is constrained by what the compiler expects, and some of it is due to the way that humans construct the code. Regardless of why it is predictable, it has been shown that code is accommodating to the same kinds of language modeling as natural language  \citep{hindle2012naturalness}. 


Recently, \cite{ray2015naturalness} showed that it is possible to predict buggy lines of code based on the entropy of the line with respect to a code language model. 


\section{Related Work}

\subsection{Bug Dection}
For software bug detection, there are two main areas of research: bug prediction, and bug localization. \\
Bug prediction, or statistical defect prediction, which is concerned with being able to predict whether or not there is a bug in a certain piece of code, has been widely studied in recent years \citep{atal2009systematic}. With the vast amount of archived repositories in websites such as github.com, there are many opportunities for finding bugs in code.\\
Static bug finders, automatically find where in code a bug is located. There has been a wide array of recent work in this area \citep{rahman2014comparing}.

\subsection{Natural Language Processing}


\section{Methods}

\subsection{Recurrent Neural Network Language Model}

\subsection{Experimental Setup}


\section{Results}


\section{Threats to Validity}

\section{Conclusion}



% In the unusual situation where you want a paper to appear in the
% references without citing it in the main text, use \nocite


\bibliography{example_paper}
\bibliographystyle{icml2015}

\end{document} 


% This document was modified from the file originally made available by
% Pat Langley and Andrea Danyluk for ICML-2K. This version was
% created by Lise Getoor and Tobias Scheffer, it was slightly modified  
% from the 2010 version by Thorsten Joachims & Johannes Fuernkranz, 
% slightly modified from the 2009 version by Kiri Wagstaff and 
% Sam Roweis's 2008 version, which is slightly modified from 
% Prasad Tadepalli's 2007 version which is a lightly 
% changed version of the previous year's version by Andrew Moore, 
% which was in turn edited from those of Kristian Kersting and 
% Codrina Lauth. Alex Smola contributed to the algorithmic style files.  
